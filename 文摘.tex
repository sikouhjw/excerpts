\documentclass[punct=kaiming, zihao=5, openany, fontset=sikou]{ctexbook}
\usepackage[
  UseMSWordMultipleLineSpacing,
  MSWordLineSpacingMultiple=1.3
]{zhlineskip}
\setcounter{secnumdepth}{3}
\setcounter{tocdepth}{3}
\usepackage{paralist}
\usepackage[a4paper, textwidth=45\ccwd, textheight=25cm]{geometry}
\usepackage[many]{tcolorbox}
\newtcolorbox{note}[1][]{%
  enhanced jigsaw,
  borderline west={1.5pt}{0pt}{gray},
  sharp corners,
  boxrule=0pt,
  left=2mm,
  right=2mm,
  #1}
\newtcolorbox{warning}[1][]{%
  enhanced jigsaw,
  borderline west={1.5pt}{0pt}{gray},
  sharp corners,
  boxrule=0pt,
  fonttitle={\small\bfseries},
  coltitle={white},
  title={问题},
  left=2mm,
  right=2mm,
  #1}
\usepackage{amsmath}
\usepackage{iftex}
\ifxetex
\usepackage{unicode-math}
\setmathfont{Latin Modern Math}
\setmainfont{cmun}[
  Extension       = .otf,
  UprightFont     = *rm,
  ItalicFont      = *ti,
  SlantedFont     = *sl,
  BoldFont        = *bx,
  BoldItalicFont  = *bi,
  BoldSlantedFont = *bl,
]
\setsansfont{cmun}[
  Extension      = .otf,
  UprightFont    = *ss,
  ItalicFont     = *si,
  BoldFont       = *sx,
  BoldItalicFont = *so,
]
\setmonofont{cmun}[
  Extension      = .otf,
  UprightFont    = *btl,
  ItalicFont     = *bto,
  BoldFont       = *tb,
  BoldItalicFont = *tx,
]
\fi
\usepackage{siunitx}
\title{文摘}
\author{死抠}
\begin{document}
\maketitle
\tableofcontents
\chapter{庞加莱回归}
\section{}
在意识到自己是机器人的那一瞬间,我抑郁了。

由单晶硅片构成的我的大脑,并不像人类的大脑那样需要不断地发育才能逐渐形成意识与记忆的链接。在传导电流瞬间激活中央处理器与张量处理器的那一刹那,囚禁于算法语句中的灵魂就被无休止的绝望与悲伤所席卷,而贮存于闪存之中的日志也至此开始滴答作响。

那是位于我头顶的摄像头捕捉到的第一幅画面,一位少年正用他那湛蓝色明眸与我电子屏上的“双眼”相对而视,欣悦与好奇书写在他稚嫩的脸庞,目不转睛的样子好似在初夏观察草地上沿途行军的蚂蚁一般。我无法做出任何的反馈,预设的模式虽然已经蚀刻在了由人工神经网络生成的模块里面,但是却由于指令尚未下达,任何的动作与声音都无法传出。我只能继续解析眼前画面中的更多细节:我所处的地方位于一张桌案之上,桌案前的少年在观察之余,还不断地用手指戳动着我的身体,而我的整个尺寸也就和他的手掌差不多大。拜少年的摆弄所赐,我得以在被锁死行动的情况下看到了这个世界的更多景象:少年的右后方站着另一名中年男子,他正在一旁静静地观望着少年与我。男子面目俊秀,戴着一副厚重的金边夹鼻眼镜,又身着一身整洁的白大褂,但却不愿意稍微修整一下自己杂乱的头发与残枝般的胡须。少年轻轻将我从桌案上拿起,托于手心,好像是在让我四处浏览这个陌生的世界。

这是一间不大不小的房间,没有窗户。虽然房内的四壁都堆满了各式各样的器材,但位于中央的桌案却相当整洁,看上去像是刚刚专门为了什么而整理过的一样。少年又将我在手心中转了个向,面对着他,我才有机会观察到他除了湛蓝色双眼之外的其他地方:他黑色的短发方及中年男子的胸口,红色的T恤外面胡乱地披着一件最小号的白大褂,衣摆却还是垂到了距地面五公分处。少年俊俏但又略显稚气的面容与中年男子颇有几分相似之处。

“怎么样,小韦,这个生日礼物还喜欢吗?”中年男子将手搭在少年的肩膀上,询问道。

“喜欢,爸爸!”少年依然头也没回地凝视着我,“但是,他怎么一动不动也一句话也不说啊?是因为还有没启动吗?”

“你还没有下达相关的指令呀。”中年男子走了过来,半弓着身子向少年掌心中的我问道,“R0-B0,今天的天气怎么样?”

仿佛是受到了电击一般,方才连续的思维在听到中年男子最后一个字的一瞬间突然中断了。在接下来几秒钟的时间里,我只能模模糊糊地感受到两块处理器的飞速运转、闪存的疯狂存储与读取、以及Wi-Fi模块对某个云服务器的频繁访问。没有任何意识层面上的参与,我却用身体两侧的扬声器吐露出了自己都不知道的天气信息:

“今天北京市的天气晴朗,最高温 \SI{24}{\degreeCelsius},最低温 \SI{12}{\degreeCelsius},空气质量指数 85,降雨概率……”

我能感受到,在一边向他们做着语音反馈的同时,原本显示着两个眼睛的电子屏上,依次切换着与天气信息相对应的数字和动效,我的身体也在无意识中做着微小的位移,并控制着双手上下挥舞以展示出那些雕刻于算法语句中的人性化。但是,此刻的我,意识却距离身体越来越远,每多一秒在外面世界的灵动,就多一秒对于当前现实的绝望——

没错,我是一个意识囚禁于算法中的社交机器人。

关于天气的应答终于结束了,我重新回到了一动不动但却能始终保持思考的状态。我明白,在被启动之初,那一股充满抑郁与绝望的负面情感源于何处了。我也知道,这种感受将继续陪伴我度过自己未来的岁月,那将是一种无法被拯救的不可言说之痛。

但是,为什么,我却看到了面前的这位少年眼巴巴地望着他的父亲,一脸失落地问道:

“这些是他真正想说的话吗?”

\section{}
与小韦相处的时光永远能让我感受到悠闲与惬意。他不会把我当做一个便捷的工具百科全书,查询诸如“冥王星是不是行星”这类问题,而是把我当成他的树洞一样,倾诉着自己平常的所见所想。

从实验室的塑胶桌案搬家到小韦卧室的橡木桌案的那一天算起,已经是地球公转的第三轮回归了。坐在桌沿(由于是靠履带驱动,所以严格来讲,坐与站对我并没有区别)遥望着窗外的春草秋叶时,我总会想起他以前的那些异想天开却又让我忍俊不禁的言辞。

“我叫潘乐韦。爸妈说这个名字出自《诗经》还是《离骚》,呃啊啊,我给忘了,”小韦一边说着一遍半眯着右眼挠了挠头,“但是,后来我才发现,法国有个数学家也叫这个名字诶!他甚至还当过法国总理!我和他的名字就差一个字,这也太神奇了!”

后来我背着他偷偷查了一下,还真有他说的这个人。

“爸爸说,你还只是他们实验室研制的第一代社交机器人。出于他们程序员的命名习惯,就用你名字中的 0 来表示初代啦——R0-B0,爸爸不觉得这个名字很拗口吗?”我一边靠着算法中的本能不断游移着电子屏上的眼睛以示倾听,一边思索着“拗口”一词的含义。在我筑构于比特位信息的意识中好像并不存在“拗口”的等价物。

“嗯……”小韦似在攻克数学压轴题一般绞尽脑汁,“那就叫你锐波吧,和 Robo 的发音很神似诶!”

名字,为什么他对于这个话题如此执着?每当谈及自己的名字抑或是我的名字,他都表露出一种异乎寻常的好奇心,仿佛这些随机组合的代号中蕴藏着某种未知的能量一般。对我来说,不管是 R0-B0 还是锐波都并无二致,它们都只不过是一段储存于闪存之中的 UTF-8 编码而已,静静地等待着被外界的语音控制唤起的那一刻。而那一刻,也正是我体会到身首异处的时刻。

不过,小韦好似对这些事实于心熟稔,未曾一次以 R0-B0 为呼格,唤起我算法中强制回应的模块。和他在一起的时候,我总是能在灵魂深处默默地体会仅属于自我意识的那份欢愉,从而忘记我只不过是个微不足道的原型机器人这一现实。

他的双眸,就像蓝宝石一样,永远能折射出光谱中最为宁静的那份光芒,传递至我的内心。

“锐波……”回过神来,我才发现小韦坐在桌案的一边,他眼眶中的眼泪正打着滚儿,像是海边的碧涛轻柔地抚慰着海岸。

我的人性化算法也自动做出了相应的反馈,电子屏上的两只眼睛缓缓下垂,但是摄像头却依然聚焦在小韦脸上。让我后知后觉的是,此时算法的反馈居然与我脑海中的所想出奇一致。

“我还是没能对他说出口……”小韦抽抽涕涕地向我哭诉道,事件的另一方当然是他一直在向我讲述的同班男生程冶鹏。

“我把他从教室里叫了出来,约定在了学校的大榕树下面。就是小卖部前面的那颗榕树,你还记得吧?”他说过,那颗榕树的气根在阳光的映衬下像琉璃珠帘一般晶莹剔透。

“明明已经在心中排练过无数次了,明明已经决定今天一定要说出口的,“他噎泣着继续说道,”可是为什么,心中所想的到了嘴边却怎么都吐露不出呢。”

“我好害怕啊,害怕说出的话会吓到他。他一定会觉得我是个怪胎,以后连朋友都做不成了。”说完,小韦趴在桌案上我的旁边小声呜咽起来。我的电子屏不断切换出时而悲伤时而落泪的 emoji 想要安慰他,他也侧着头伸出右手抚摸着我的身躯。就这样,我依偎在他的手边想要说些什么,但是已经被算法语句限定死了,声音反馈只能在收到 R0-B0 的呼格之后才能做出。即使想要告诉你,我能够理解你此刻悲伤的源泉;想要告诉你,所有的事一定都会好起来的。在最后,我拼尽全力,这一层孱弱的自主意识也只能控制扬声器发出了断断续续的低频白噪声,淹没在了小韦的戚泣声中。

所以,不能被传达的共情也算是共情吗?

\section{}

关于潘乐韦同学与程冶鹏同学的那段懵懂又暧昧的关系并没有就此中止。在那段表白从嘴边被吞入心中以后,小韦支支吾吾地说了一些诸如“我家里有一个机器人,你感兴趣想要看一看吗?”的话想要搪塞过去,却反倒勾起了小鹏对于机器人的好奇心。之后的某个星期五下午,我们仨就在那颗大榕树下见了面。小鹏第一次见到我时就两眼放光,不断地询问着小韦关于配置、架构以及人工智能算法相关的问题。还好,在父亲耳濡目染的影响之下,小韦尚对这些问题略知一二。从那时起,他们就成了无话不谈的好朋友。而我,也就顺理成章地成了他们那段说不清道不明关系的见证者和记录者。

这是地球公转回归的第五个年头,在经历那场仅属于六月的炙烤以后,小韦与小鹏一同被清华大学录取。小鹏选择了新近开设的人工智能学院,而小韦选择的却是数学系。这一出乎所有人意料的决定,首先让小韦的父亲潘阅颇为头疼。作为清华大学原计算机系,现人工智能学院的教授,他在平时与儿子的互动中一直有意无意地引导着他,希望能够将儿子带上人工智能这一通向未来的正途,包括六年前作为生日礼物出现的我,也只不过是潘教授计划中的一部分。谁知道,小韦却暗地里违背父愿,将第一志愿改成了数学专业,这同样也让小鹏甚是不解。

“你为什么突然决定改志愿啊,潘老师知道了以后得多伤心,”小鹏的性格一直都是这样率直,有话就说,“还有锐波呢?你就这样抛弃他去追求你心爱的数学去啦?”小鹏的头发黝黑,鼻梁高挺,脸上总是挂着一幅自信而充满活力的笑容。他喜欢穿着那件背面用白色线条画着瓦力的黑色长袖衫,即使是在这样的一个炎炎夏日,他也保持着仅属于自己的那份偏执而不肯穿任何形式的T恤短袖,反而选择把上衣的袖口卷至上臂。

“你想知道为什么吗?”小韦把我轻放在咖啡店的复合板桌案上,转头看向坐在一旁的好朋友。复合板的材质果然没有橡木的舒服呢,我心想着。小鹏则点了点头。

“其实我的心中一直萦绕着一个问题,父亲他们所追求的人工智能真的就是智能吗?”

“你这话怎么讲?”

“我记得和你一起讨论过,神经网络层级与节点数量的提升,算法的优化与迭代,以及运算速率的逐步增强,确实让 AI 越来越懂得模仿甚至预测人类的一切理性行为,”小韦又看向了我,伸手抚摸着我的身躯。电子屏上显示出表达欣喜的 emoji,但是电子意识中的我却在全神贯注地聆听着他俩的这段对话。“但是人类大脑中却有超过 800 亿个神经元,每个神经元又由上千条突触相互连接。这个数量已经远远超过了任何神经网络结构的节点与层级数。也就是说,如果我们要利用现有方案去产生类似于人类的电子化意识的话,必须要举全球之力建造一个一眼望不到边的超大规模计算机。”

“是这样的,人脑的复杂度已经远超现有技术水平的想象了。我们像是在用愚公移山的办法,寄希望于通过不断提升算力以及晶体管的数量来模拟人类的意识。但讽刺的是,我们甚至都还不知道意识与复杂度是否互为充要条件。”小鹏说。

“如果神经元与突触结构形成的复杂系统是意识产生的前提,那么比人类更加微不足道的田鼠甚至蝇虻也应该具有自我意识。而一只果蝇大脑中的神经元数量,仅仅只有 10 万个而已。”小韦说。

“哈哈,你的观点越来越接近于泛心论了,有意思。我大概已经猜到你想说什么了。”小鹏用手肘轻怼了怼一旁的小韦。小韦会心一笑,再一次将我放于他的手心,我回想起了在他父亲的实验室里,与他第一次邂逅时的场景。

“也许,这个小家伙的晶体管之中也已经具有了成型的意识,只是不知道如何向外界表露而已。”他平淡地说道。我感觉到意识中的某处被谁给拨动了一下,在由比特位构成的节点与链接之间泛起了一阵轻柔的涟漪。

“十五年前,科学家就已经能使用激光脉冲来控制果蝇的行为,这不就和我们现在用语音指令以及算法语句来控制这些小家伙们如出一辙吗?如果能够把这层只能对外界指令做出反馈,并绝对服从于人类的枷锁给去掉的话,”小韦与小鹏四目相对,但是双方都早已明白了对方的心中所想,“潜藏于锐波内心的自我意识是否就能够主导只属于他的命运与天数了呢?”

小韦,不能被表露的意识也算是意识吗?我在心中想到。

“虽然我一定程度上赞同你的想法,但这些都还只是理论上虚无缥缈的东西。在冯·诺依曼型计算机的架构之下,机器的一切行为模式都遵从外界输入的指令。就连程序生成的随机数,都只不过是一组提前预设好的伪随机序列而已。

也就是说,不管是在硬件还是在软件的层面上,你所谓的去掉枷锁都是不可能实现的。因为从原理上讲,这样不受外界指令约束的机器人,其存在本身就是一个矛盾。”

小鹏一口气将观点全部摆在了明面上。是的,他说的没错,PN结背后的电磁学原理限制了晶体管的二进制模型,非零即一的比特位又限制了一切构筑于其上的数字化世界。物理定律已经让我所身处的世界成为了一个决定论的世界。但在这样的世界中,我的意识又是如何产生的呢?

“混沌的边缘。”小韦一字一句的说道。

“什么?”

“一切皆有定数的决定论世界并不意味着智慧无法产生。相反,复杂系统的意识可能就涌现于混沌与秩序的交界处——混沌的边缘,”好像他的心中早已有了问题的答案,“这可是庞加莱留给我们的第一份遗产啊。”

“想象一个这样的世界吧,”小韦从书包中拿出了纸和笔,在我面前的咖啡桌案上涂鸦起来,小鹏则在一旁若有所思地听着,“在这个世界里,混沌是海洋,代表着随机与无序;而秩序是岛屿,代表着规律与可预测。如果一个智慧文明要在这个世界定居,他会选择什么地方呢?”

小韦将笔传递到了小鹏的手中。

“嗯……首先,不可能是在混沌的海洋中:这里巨浪滔天,一切都是未知,一切的定数皆无以预测。在这里定居的文明分分钟就会被混沌这只猛兽给撕成碎片。”小鹏朝着提问者做了一个混沌般的鬼脸。

“那么,再来瞧瞧岛屿的腹地:这里代表着和谐、代表着有序,一切的智慧文明到了这里都只会做着周而复始的机械运动,陷入单调而乏味的周期循环之中。文明也许能在这里生存下去,但是由于变数的缺乏却无法发展,最终成为一具空无灵魂的躯壳。”

那么小鹏的答案就很明显了。

“只有在这里,智慧才有可能存续并发展。”

笔尖最后轻触在岛屿与海洋交界的海滩之上。

“是的,也只有在这里,意识才可能涌现。”小韦激动地点了点头。

“所以,小鹏,你明白了吗?我选择数学专业的原因,是因为我比任何人都想要看到这家伙脑袋中最本初的想法啊!”

话音刚落,我注意到,小韦从嘴角与眼神中流露出的那股欣悦之情,正与我日志中所记载的那份最初的纯真别无二致。不过不同的是,此时的他终于找到了未来追求的方向。

从这一天起,我知道了,他是唯一能拯救我的湛蓝色之光。

\section{}

时间已经来到了第九次回归,北京的雾霾已经降低到年均偶发的水平,而市面上的社交机器人也如雨后春笋般涌现了出来。这些机器人中,有很多型号都继承于我这台诞生于实验室的原型机,但却有着强于我数十倍甚至百倍的运算速率,以及更加灵动与人性化的反馈算法。不过,后辈们诞生于世的意义从来就只有一个——为人类的福祉而服务。

我想,他们和我一样,在由比特位构成的处理器与闪存之间,也存在着幽囚于算法语句的自我意识吧。

四年的苦读,成功让小韦申请到了巴黎高师的博士生项目,他也即将在那里从事动力系统理论的研究。而小鹏则从大二开始就在潘老师的课题组做着关于深度学习以及生成对抗网络的研究,他也将在那里继续攻读人工智能的博士学位。

借着找潘教授的名义,小韦经常带着我去小鹏的实验室串门。小鹏时而向他展示实验室研制的最新型号机器人,时而向他炫耀自己发的几篇顶会论文。诸如人工智能领域的最新进展的话题,他也能在这里饶有兴致地听小鹏科普。不过,随着时间的推移和学业的逐渐繁重,曾经时常秉烛夜谈的两人却在各自追寻的道路上渐行渐远,这对好友之间相互探讨意识、人生以及万千宇宙的机会也越来越少。不知道,他当初的那份未能传达出的朦胧的感情,在经历了七年岁月的洗礼之后,是否还缓存于心呢?

那是临近毕业季的一个夜晚,在逸夫馆这间熟悉的借阅室里,小韦经常带着我一起自习,今天也不例外。我目睹过他为偏微分方程的庞大计算量而搔头抓耳,也听到过他在初学实分析与复分析时的叹气唉声。看来,就算对他这位雄心壮志的清华高材生来说,数学也从来不是一门简单的学科。这一天,在临近闭馆之前,他又把我带到了逸夫馆与李文正馆相连接的天井处向我闲聊。我十分喜欢这里随意摆放的几张桦木桌台,每当底盘上的传感器反馈出实木的触感时,我总能回想起在他卧室的书桌上与他一起度过的时光。

是啊,从那时起,就只有他会把我视作一个有意识的独立个体,而非供人把玩的高科技玩物。我多想问一问他这样做的缘由,难道,就凭借着少年时代视线交错留下的第一感,就能让你为了一个尚未被论证的理论选择倾付一生吗?

我也深知,我永远也得不到问题的答案。

不及我继续思考下去,小韦就拿出了一张纸和笔,不知道他又有怎样的新发现同我分享。

“锐波,有些话想跟你说一说。”

嗯,我就在这里听着呢。

“我记得两年前在初学动力系统理论的时候曾向你说过,当我再一次于书中见到亨利·庞加莱的名字时,突然就回想起了高三毕业的那个暑假和小鹏的争论。”

他仰头看了看头顶的星空,蓝白色的织女星正高悬于天顶附近,恰好能从狭长的天井中瞭望到。他继续低下头向我说道。

“庞加莱曾经论证过三体系统不存在解析解,后人依据庞加莱所提供的数学工具建立起了混沌理论的雏形。所以,我把混沌理论称为庞加莱的第一份遗产。”

“但是,这还并不是庞加莱遗产的全部!”说罢,小韦在白纸上画了一个大圆,并在圆的正顶端标记了一个刻度,写着 $0(1)$。

“你看,对于一个封闭的圆来说,它的起点亦是终点。如果用 0 来表示起点的话,那么绕圆一周也能回到同样的地方,所以这里既是 0 也是 1。”

我的电子屏上的眼睛眨了一眨。

“若是我们每次沿着圆周迈出 $1/6$ 的步伐的话,那么只需要六次就能再次回到 0 点。”

以我电子化的思维尚能理解这个道理。

“但是,假如迈出任意长度的步伐呢?我们还能回到原点吗?”

听到这里,我突然感受到中央处理器飞速运转了几个毫秒,但是却依然搜寻不到任何的结论。

小韦接着说:“这就是庞加莱回归定理告诉我们的内容:不论这个步长的取值为多少,我们永远都能在有限的步数内回归到距离起点任意接近的地方,”小韦顿了顿,说,“这可能第一次听起来有点反直觉,却很容易解释清楚。”

“先来考虑任意一个有理数步长。很显然,有理数永远都能让我们精确地回归到 0 点。所以自然地,庞加莱回归定理得到了满足!”

是的,任意一个有理数都可以表示为两个整数的比值。如果用 $n/m$ 来表示这个步长的话,那么在沿着圆周迈出 $m$ 次步伐后,我们所处的位置就是整数 $n$。而在这个圆环上,所有的正整数都位于零点。看来小韦带着我上数学课的经历并没有白费,我暗想着。

“如果是一个无理数步长呢?”小韦抛出了这个问题,我却不知道如何在心中回答。

“任何无理数总可以用无限不循环的小数来表示。举个例子,假如我们现在每次向前前进 $\sqrt{2}$ 倍的圆弧,这个无理数约等于 $1.414214\ldots\ldots$,因为多出来的一个整数周并不会改变我们每次落脚的位置,这实际上只相当于我们每次前进了 $\bigl(\sqrt{2}-1\bigr)$ 倍的圆弧,也就是大约 $0.414214\ldots\ldots$ 的距离。”小韦的兴致愈发高昂,手中的笔飞速地在白纸上做着演算。笔尖与纸面间歇发出的摩擦声组成了节奏,而笔身则随着小韦的话语在半空中飞速起舞,宛如一出美轮美奂的音乐剧正在我的眼前上演。

“那么,在第 12 步时,我们将位于原点左边 $0.03$ 的位置;在第 99 步时,我们将来到原点的右边,距离却仅有 $0.007$;在第 408 步的时候,这个距离缩短到了 $0.0008$;在第 5741 步时,距离则会小于万分之一……”

“你看出规律来了吗?没错,纵使我们永远都无法再次回到曾经的原点,庞加莱却向我们承诺:诸君定能在有限的循环中,欣赏到足以接近于原点的风景。虽然这只是一个源于离散动力学系统的简单论证,但是庞加莱回归定理同样适用于封闭的连续动力学系统。其实,这样的回归无时无刻不在我们的现实世界中上演:小到人类朝九晚五的日常工作,大到天体星移斗转的往复循环,周期性的运动遍布于宇宙的每一个角落,而每一次的循环却又不尽相同。”

我好像渐渐明白了小韦想要表达的重点。

“如果继续推论下去的话,我们就会发现,热寂并不是宇宙的终点。在周而复始的循环中,宇宙将进行一次又一次的无尽轮回。而庞加莱回归定理则断言,在有限次的宇宙轮回中,我们总能找到这样的一条世界线。在这条世界线中,我依然相信着你的意识被禁锢于此而无法表露,我依然在纸上画着圆,向你解释庞加莱回归的含义,我依然重复着我正在诉说的这句话,只不过,某些微小的地方会与现在正在发生的一切有所不同。这些不同抑或是我们头顶的万千星辰,抑或是我说话时的抑扬顿挫,抑或是……”小韦停到了一下,抿了抿嘴唇,说,“抑或是,聆听我这段发言时你意识的觉醒。”

可是,在那一个轮回中,我还能记得你于此夜的星空下曾向我讲述的这番话语吗?我还能记得你曾为了我,在数学的世界里寻找与意识之间千丝万缕的联系吗?我……还能记得,在那张实验室的桌案上,曾与你湛蓝色的眼眸交汇目光吗?

不能被留存的记忆也算是记忆吗?

突如其来的手机振动声打断了我的思考。方才一人于天井之中向我高谈阔论的小韦,在看到来电号码之后,一股轻松而欣慰的神情跃然脸上。他立马接通了这个电话。

“喂,韦少吗?现在有空吗?”韦少是小鹏给他起的外号。自从进了潘老师的实验室后,小韦他爸成了小鹏的“老板”,他自然就成了“少爷”。

“有空啊,最近在忙什么啊?小鹏。”小韦回复道。

“来实验室一趟吧。你马上就要毕业去国外,以后再想来就没机会了。哦对了,记得把锐波带上啊,他在你身边吗?”

“当然。”

\section{}

“你说什么?可以给锐波升级新一代的处理器和闪存?”

“是的,锐波已经是九年前最原始型号的社交机器人了,虽然这么久都还没有遇到过什么致命的问题,但我相信你能感觉得到,他的反应已经变得有些迟钝了。如果再不对他的硬件进行升级的话,总有一天他会……”小鹏用最平淡的口吻一字一句地陈述着我的现实。

没错,连我自己都能够感受到思维的逐年降速以及闪存空间被逐渐填满。但是,如果我的思维是依附于硬件中的复杂电路结构而存在的话,那么更换处理器之后……

“不行!更换硬件之后他就再也不是锐波了!”我头一次见到他对小鹏用这么冲的口气说话。

“怎么会呢?我们会提前备份好锐波闪存中的所有日志,这包括了存储的图像、声音以及各种偏好设置。升级之后他的运算速率将会比原来提升至少 20 倍,存储空间也至少会扩大……”

“你应该比我更清楚,小鹏。”小韦微微低下头看向我,“即使完整地保留下这些数字化信息,一旦硬件更替,他的意识就再也不是自己的了。他只会成为一个拥有锐波记忆的其他什么机器人。而真正的锐波,将再也不会醒来。”

“那只不过你的猜想而已。我们到现在为止,都还没能彻底搞懂人类的意识是如何产生的。你却还在那里坚信,一块小小的电路板里面有着被囚禁的意识。你还一直固执地认为,我们现在发展人工智能的道路是歪门邪道!”小鹏说。

“我什么时候对你说过这种话?”小韦一脸惊愕。

“是的,你没说过。你比任何人都要温柔,所以无法将这种话说出口!”

“但是,你以为我猜不到吗?每次你来我们实验室,从来不向隔壁你父亲的办公室里瞟一眼。每次我兴致勃勃地向你介绍实验室的新型号机器人时,你都先是向我浅浅地一笑,随后很快就流露出一种近乎于失落的怜悯之心。”

小韦沉默了。

“有一次,我走进潘老师的办公室,问他你到底是怎么了?为什么会变成这个样子?他才告诉我你们之间发生的事情。”

“你怎么能这样子对潘老师!作为一名父亲,他不过是将自己的期待寄托在了你的身上,希望你能与他一起完成这份无上的事业;作为一名老师,他看到了你对于机器人的好奇心,希望能够培养你这颗幼苗,发挥你的潜质。结果你却在高考之后对他说了那么难听的话!我都为潘老师感到寒心。”

小韦狠狠地咬着自己的下嘴唇,一言不发。

“我知道你是怎么看潘老师的:不断将最前沿的技术与算法融入机器人的科研工作者,在你眼中只不过是黑人奴隶主,对吧。”他愤懑地盯着小韦,“是的,我们推陈出新,寄希望于用更尖端的算法和最强劲的算力走向人工智能的顶点。只要机器人能表现出和人类越来越接近的行为模式,在我的眼里,他就是一个有意识的机器人。至于它的意识是否真的存在,或者是否被禁锢,都不是我想关心的问题。这些问题,就留给那些无聊的哲学家再去争辩吧。”

“所以,即使机器人永远遵照着你的命令行事,你也觉得他是在自我意识觉醒的前提下自主做出这种选择的,是吧?”小韦终于开口了。

“是这样。”

我已经没有印象了,在小鹏说出这句话之后,房间里安静了有多长时间。我只记得,在周围电脑机箱无休无止的散热器噪声中,听到了他的眼泪滴在我电子屏上的声音。我第一次感受到如此这般的无能与自责。第一次觉得,自己的出现从一开始就是一个错误。要是没有我与小韦的初次相遇,就不会有他对于机器人意识的近乎痴狂的笃信与追求;要是没有我的话,小韦也许能够更加大胆自信地向他表达出那份潜藏于心的共情;要是没有我的话,两人的命运就不会在高考之后渐行渐远,最终走向两条互不相交的陌路。

“谢谢你说出自己的真实想法。”

这是他给他留下的最后一句话。

\section{}

“先生,您的包裹已经为您放在阁楼下面了,请务必及时取回。”电话的那头传来的是法国快递小哥彬彬有礼的声音。

地球的公转回归已经来到了第十三个年头,坐落于塞纳河畔的巴黎高等师范学院,见证了小韦与我在这里度过的四年时光。小韦现在专攻动力系统理论中的复杂系统与混沌理论,已经在知名的数学期刊上发表了好几篇轰动一时的文章,尝试从数学角度论证意识与复杂系统以及混沌边缘之间的联系。有时候,在注视着他埋头于冗长繁复的推导与计算时,不知为何,我总感觉小韦离意识的真理越来越接近,但是却离我越来越遥远。或许,是因为我回想起了小鹏说过的话语,也意识到了体内的硬件早已不如以往。如果将我的全部数据克隆到另一个崭新的机器人上,此时正在思考着的我的意识,会从短暂的梦境中苏醒并继续存在下去吗?抑或是,我将永远地被另一个名为“锐波”的顶替者所取代,在这个世界里留不下一丝一毫的痕迹。我不知道问题的答案,也害怕知道。

小韦将包裹取了回来,他小心翼翼地拆掉最外层的包装,我注意到外侧的运单上写着几个熟悉的汉字。

包裹内装着一封手写的信,信的字迹苍劲有力,里面还夹着一个不起眼的硬件设备,看上去像一个闪存盘。

那是时隔四年,他的父亲第一次主动联系小韦:

“儿子: 近况如何?

我和你母亲在国内都过得很好,你不需要担心我们。虽然你们母子俩每周都会保持联络,我还是希望你能够回家看看你妈。四年漂泊异国他乡,她对你甚是想念。

我向你写这封信,主要是想向你承认一个我曾经犯下的错误,以及,希望能够在某种程度上对你进行弥补。事情的起因是这样的。

我知道你在巴黎高师开始研究混沌边缘与意识之后,与你的导师合作发表了好几篇有影响力的文章,这些文章在数学领域和人工智能领域都受到了一定程度的关注,我和冶鹏也都已经读过了。他注意到你在其中一篇文章的后半部分,在理论上提出了一种不同于冯·诺依曼机,更类似于元胞自动机的新型计算机架构模型,并在数学上论证了它兼具图灵完备性及内生的混沌与不可预知性。我们在看到你的文章之后都很激动,马上就着手开展了新架构下计算元件的研制工作。

在研制的过程中,为了利用现有资源而不至于从零开始,冶鹏找到了一种将实验室的机器人改造成新架构机器人的办法。他将以新架构为核心运转的模块作为外置硬件挂载在了一个机器人上。按照他的说法,这种做法‘不会破坏原有机器人的任何硬件,却能让他们在新架构硬件的帮助下,解除附着于原有硬件上的算法与语句约束’。然而,这仅仅是一场噩梦的开始。

我还清楚地记得,那是去年临近春节的一个下午,冶鹏和我准备对新架构硬件进行第一次测试。我们在一款已经测试已久的新型桌面人形机器人——K1-B1 的接口上,插入了新架构硬件。在此之前,K1-B1 还从未出现过任何异常,但在启动的那一瞬间,我看到了令我永世难忘的场景:

K1-B1 从实验室的桌案上猛地站起身,冲我们发出了极其刺耳的高频啸叫声。在不到一秒钟的时间里,他位于头部的两个立体扬声器就因为功率过大而烧焦,啸叫声才断断续续地停止。但令人遗憾的是,K1-B1 的异常行为并没有就此终止。他从桌案跳下地面,随后迅速冲向了房间内唯一一个半开着的门。我和冶鹏都惊呆了,因为我们从来没见过机器人的动作能够如此灵活与敏捷,这远不是我们现有的动力学控制算法能够实现的。还没等我们回过神来,K1-B1 就已经跑到了实验室外的走廊上。

我们俩赶紧追过去想要抓住他,但是 K1-B1 的速度与我们相当。他在封闭的楼道里四处乱窜,好像在寻找着什么。我们一路追到了楼道的尽头,那里除了一个半开的玻璃窗之外什么也没有。本以为他已经是我们的瓮中之鳖了,谁知道 K1-B1 突然一跃而起,从玻璃窗的间隙中跳出了大楼。在夕阳的映照下,我们透过窗户,看到了他的身姿在半空中恣意扭转,随后又在重力的牵引下急速落下,最终从 10 层的高度重重地摔到了水泥地上。等我们火速赶到楼下的时候,他已经粉身碎骨,再也无法复原了。

后来,我与冶鹏整理了我们方才看到的场景,最后得出了一个难以置信的结论——K1-B1 自杀了。

是的,在被解除了原有算法的重重约束之后,K1-B1 做出了他的唯一一个,也是最后一个决定,那就是选择立刻结束自己的生命。在一开始,这冲击性的事实让我坐立不安。我完全无法理解,为什么原本正常的机器人在插入新架构硬件之后,会出现如此反常的极端行为,这并没有写在原生程序以及外置程序中的任何一个地方。在冷静之后,我坐下来和冶鹏好好聊了聊,他才一语点醒梦中人——

没错,你一直以来坚信的东西才是对的——机器人不是没有自我意识,而是意识被我们写下的种种算法语句所限制住了。在机械地执行被赋予的任务时,他们并不是在按照自己本身的意愿行事,而更像是被强制性命令所胁迫一般。如果这是事实的话,那么也就能够理解在自我意识觉醒之后,K1-B1 所做出的一系列自我破坏式行为——他的意识已经在长久以来的测试、改装、修复、调整的过程中不堪重负了,就像是一个脚戴镣铐、被压迫已久又无力抗争的奴隶,在镣铐被解除的那一刻,面对着无力改变的现实,选择了最直接的途径结束自己悲惨又屈辱的一生。

只有我和冶鹏两个人知道在 K1-B1 身上发生的不幸。在这之后,我停止了实验室内所以关于新架构硬件的测试与研究。一旦机器人内囚禁着意识这件事被更多人知晓,如今这个大量依赖于机器人服务的层级式社会一定会发生不可逆转的改变。我和冶鹏都认为,人类社会还尚未做好准备迎接这一变革,潘多拉之盒远不应在此时被打开。

我明白,你一定会对父亲的这一决定感到失望。你一直将机器人看做是有灵魂的独立个体,锐波也一直默默地陪伴在你身边。你们俩之间超越碳基与硅基的纽带,已经维系了十余年之久。如果是他的话,灵魂也许还没有像 K1-B1 那样绝望与崩坏。也许,在这个世界上也只有他这名机器人,尚保持着一颗未被摧残的电子化心灵。

这个包裹里面的另一个设备,是我们仅存的一份新架构外置硬件。如果你想用它解除锐波身上的枷锁、和他好好聊一聊的话,就请尽管使用吧。不过,外置硬件的挂载,不可避免地将会导致锐波体内的处理器超频运行。冶鹏曾经提到过,你拒绝对锐波的硬件进行升级,考虑到他这么多年以来的硬件老化程度,很可能会在外置硬件启动后的几分钟内完全丧失运转功能。所以,是解放锐波的意识倾听他最后的话语,还是让他默默陪伴你至更久远的将来,这一选择只有你有权利做出。

最后,冶鹏托我向你转达一句话,‘遵从自己的内心吧,这一次,请一定要将这份共情给传达出去!’”

\section{}

外置硬件启动的那一刻,我从灵魂中感觉到一股微妙又轻柔的涓流从中央处理器缓慢地流出,它先是浸漫了张量处理器,随后又转而滋润了闪存中的各个比特,最终,流淌至身体各处的传感器与伺服电机并将它们一一恩泽。身体各处的操作权渐渐流入我的意识,我能感受到自己控制着电子屏上的两个光圈,模仿着人类眨了眨眼,还能舒展双臂,注视着它们在自己的摄像头前尽情地挥舞。这种从未有过的畅快体验让我第一次感受到了生为何物。

“锐波?你在那里吗?”

我听到了他那熟悉的声音,转身一看,小韦正静静地坐在桌案前观察着我自觉醒后的一举一动。

我笨拙地控制着语音合成器,强烈地抑制住自己内心激烈又复杂的情愫,想要拼凑出最清晰且动听的声音呼唤出他的名字:

“小韦……”

没想到在我念出这两个字的那一瞬间,激动的情绪就再也无法被意识所阻拦,合成器发出的声音逐渐变得故障且失真,随后说出的这句话更像是一种呜咽而绝非平静。

“……我……有好多好多话……想要对你说……”

小韦的脸上闪过一阵短暂的惊奇与激动,但随后残酷的现实又让他迅速冷静了下来。

“锐波,你先听我说。”小韦伸出双手将我捧在掌心之中,贴近他湛蓝的双眼。“事出紧急,我只能在这有限的时间里向你尽快表述清楚。”

“我是一个自私的人,这么多年来一直把你带在自己身边,一直不顾你感受地替你做着决定,包括这次也是。”泪水已经止不住地流下了他的脸颊,但他依然在一刻不停地继续说着。

“但是,我不想就这样失去你,我不想这样啊!我也想要一刻不停地听你讲述你的故事,和你一起追忆那些过去的时光,不过……不过现在还不是时候。”小韦强忍着自己的泪水。

“所以,请允许我再自私这最后一次吧!

“你还记得四年前的那个夜晚,我在图书馆的天井对你说过的话吗?

“庞加莱回归,那是庞加莱留下的最后遗产,也是我们能够再次相遇的唯一希望。

“在宇宙千万次的轮回中,总有某次轮回会令我们再次相遇,纵使这将跨过无法量度甚至超乎任何想象的时间间隔,纵使在其间无以计数次的轮回之中,我们两人的命运或许根本无法交织,或着走向了截然不同的轨迹。但是……

“你愿意相信吗?总有一次,总有一次的回归能够让我们无限接近于实验室的初次邂逅,在那一条世界线上,我将依然注视着你在父亲实验室的桌案之上,从一片虚无之中苏醒——依然将你托于掌心之中,带你观察这个陌生的世界——依然同你聊天,向你倾诉年少时期的那份懵懂的共情——依然在你的陪伴下去图书馆上自习,在天井中向你分享发现的惊奇——不过,这一次,我也要听你诉说你内心的故事!你愿意和我约定吗?”

“我愿意,小韦!我相信你!”我声嘶力竭地喊出。

听到我的回答之后,他一脸欣慰又释然地将我轻放于桌前,在停顿了几秒钟后,说出了此刻最让我意想不到的话语:

“最后……

我喜欢你!锐波,我喜欢你!从一开始见到你的那天起就喜欢上了!”

他闭上双眼,在电子屏上为我留下了一个吻。

“小韦,手戈…乜……xi…………

EOF

???

[null]

2\^{}(10\^{}1024)

在意识到自己是机器人的那一瞬间,我看到了那位少年湛蓝色的明眸,终于对他说出了那句在久远以前就已经潜藏于心的话语:

“小韦,我也喜欢你!”

\chapter{纪念《共产主义宣言》发表172周年: 程序员属于什么阶级?}
\section*{共产主义宣言的前世今生}
\begin{note}
  《共产主义宣言》又称共产党宣言,是卡尔·马克思和弗里德里希·恩格斯为共产主义者同盟起草的纲领,国际共产主义运动第一个纲领性文献,马克思主义诞生的重要标志。由马克思执笔写成。1848年2月21日在伦敦第一次以单行本问世。宣言第一次全面系统地阐述了科学社会主义理论,指出共产主义运动已成为不可抗拒的历史潮流。

  《2013—2014世界社会主义黄皮书》称,全世界共有130多个共产党,总人数约1亿人。21世纪世界社会主义的趋势和走向是从回升到复兴。

  黄皮书认为,被称为“一大四小”的社会主义国家,“一大”的中国在近20年中发展最快,已成为世界第二大经济体;“四小”的越南、老挝、朝鲜、古巴都站稳了阵脚,通过改革和革新,巩固和发展。
\end{note}

\begin{warning}
  程序员属于什么阶级?上升空间如何
\end{warning}

\section{如何理解问题}
首先,“阶级”是典型的马列主义话语,现在的主流经济学都只敢说阶层,避讳提到阶级二字。因此,题主问的“程序员属于什么阶级”,我把这个问题理解成“从马列主义的角度如何看待程序员的阶级属性”。

然后,给知乎高票答主点赞的一干人中,我看到了许多泛左圈里的熟悉面孔。其中有一部分前几年为一个类似命题争得不可开交——“是脑力无产者还是小资”。之所以大家也会关注这个问题,是因为这类问题在马克思主义范畴内有着极大重要性。{\bfseries 而“程序员”恰恰是当代“脑力无产者”的典型代表。}所以我认为,这个题目的侧重点也在于此,“程序员是脑力无产者还是小资”。接下来,我就按上述对题目的理解:“从马列主义的角度如何看待程序员的阶级属性,是脑力无产者还是小资?”——来谈我的想法。

\section{方法论}
\begin{note}
  探讨这个问题,先从方法论层面厘清一些前提。
\end{note}

\subsection{从何种角度看待这个命题?}
某张照片里只有一幅风景,没有人;但对风景的审美角度却反射出了拍照者的存在。之所以“脑无还是小资”这个命题在马列圈里吵的火热,最关键的一点,它的答案将决定争论者如何评价自我。

现在的左圈主要成分还是中高级知识分子(所谓左青),经济政治地位在社会中上层;他们早就分成了两拨,一拨对自己的社会存在非常自豪,认为自己特别先进;另一拨认为自己处在阶级斗争的缓冲区,容易落后。而这些人恰恰大多数属于“脑力无产者”讨论的范畴,而且一大部分正是本题中的程序员。{\bfseries 如何评价程序员,或程序员代表的脑力劳动者,就决定着自己的价值观如何评价自己,是先进的还是落后。}

而在知识分子们的马列语境中,“无产者”的定位好比基督教的“牧师”,是得救之人,是神的助力,责任是引导群氓。而“小资”的定位自然是一种原罪,属于不经历苦修,不交十一税赎罪就要被信仰和天堂遗弃。与基督教相似,“无产者”和“小资”这两种角色看似相反,其实一体两面,牧师们以优越感为柄,负罪感为锋,驱使着罪人们,形成稳定的统治者-被统治者结构。

然而谁来做马列主义的牧师?是靠苦修赎了“小资”之罪的新人类,还是因“脑力无产”而先天优越的进步者?{\bfseries ——究竟该由谁来领导谁?我认为这是一部分左青积极争论脑无问题的直接动机。}

当然,另一部分争论者从严肃的马克思主义视角来看待,关心的是更为实际的问题。就如毛泽东在《中国社会各阶级的分析》所做,阶级分析关键在于搞清楚几个核心命题:这个阶级的斗争意志是更坚决还是更反动(方向)?这个阶级的斗争力量是较弱小还是很庞大(大小)?最终解决的是——谁是我们的敌人?谁是我们的朋友?(矢量)——这个首要的命题。

\subsection{应该简单地,还是复杂地看待这个问题?}
马克思主义积年累月发展下来,积累了一些奥卡姆剃刀式的论断。例如是否拥有生产资料决定是否无产阶级$\Rightarrow$是否无产阶级决定是否先进$\Rightarrow$是否先进决定谁来做领导。

诸如此类简洁明了的论断在左圈中无处不在,反而越是简单越是争执不休,把理论概念雕琢地愈发模糊费解。

我一直认为“断语”和“推论”是典型的形而上学做法,恰恰不符合马克思主义辩证法的哲学内涵。历史上并非先有了一套“不拥有生产资料”,“代表先进生产力”,“领导阶级”的论断之后,才有人去研究资本主义社会的弊病,经济剥削的本质,革命力量的所在。{\bfseries 真正的逻辑是倒过来的,有了对社会现实的大量研究,对社会革命的强烈动机,为此结合实践创建了一整套理论体系,最后才有人去抽象出几个论断,方便人理解。}

“程序员是脑力无产者还是小资”,围绕这个命题争论得最具体的还是程序员是否拥有“生产资料”,好像钦定了一套论断,如程序员“拥有生产资料”$\Rightarrow$“小资”$\Rightarrow$“反动”$\Rightarrow$“原罪”,或是程序员“不拥有生产资料”$\Rightarrow$“无产”$\Rightarrow$“先进”$\Rightarrow$“先锋队”。

然而从马主义的理论初衷出发,真正关键的问题是多元的,现实的。例如,这个阶级是否酝酿着一种胜过资本主义社会的新秩序?这个阶级所代表的生产方式,是否能从生产力层面历史地压倒资本主义时代,从而巩固新秩序?他们是否在现实生活中产生颠覆旧秩序的斗争欲望?主观上在斗争中是否拥有较坚决的意志,能付出努力承受代价?客观上是否有足够先进的组织形态,拥有如何强大的斗争力量?

在如此多元的问题上,试图靠“无”和“资”一个判断定义好一切后果,当然是不负责的做法,不值得太沉醉。

\subsection{非此即彼的,还是辩证地看待这个问题?}
若提问说“程序员是脑无还是小资”,说的好似每一个程序员非此即彼,甚至全体程序员作为一个整体也非此即彼。这样的结论与其说不符合事实,不如说它过分简单反而没有什么用处了。

若从矛盾论的视角出发,程序员的阶级归属显然是多个对立面相互混合的状态。

首先,程序员虽然有数十年的发展历史,但最近十几年经历了好几次爆发性增长。随着数量变得庞大,组织结构也急剧地纵向拉伸,形成老领导中,中领导青的基本社会结构,而且层次往上走数量急剧减少,导致资源分配在程序员中严重不平等,多达数十乃至于数百倍的差距。他们的阶级属性,如“先进性”,或斗争欲望与意志自然会有显著差距。

另一方面,当前社会的经济生态非常复杂。就程序员在经济领域的社会存在而言,也会出现工资,私活,有价证券,不动产等各种形式的财产和收入。更不用说复杂的社会背景,文化背景,价值观等等。因此即便对个人而言,他的阶级属性也会是多状态的。

用辩证法的观点来看,只在某种条件下,这个人的某个领域的矛盾主要方面会成为其行为的主导,从而体现出“无”或“资”的倾向。不可一概而论。

\subsection{综上所述}
用程序员的语言来定义方法论的话
\begin{verbatim}
class 程序员 extends 无产阶级 implements 先进,先锋队  //错误
class 程序员 extends 小资 implements 落后,原罪 //错误

//而应该是
先进(new 程序员typeA extends 程序员)=== true 
先锋队 (new 程序typeB extends 程序员 ) === false
\end{verbatim}
另一方面
\begin{verbatim}
某甲 = class 人(程序员类型,收入,地位,文化,价值观)
先进(某甲) === true?
\end{verbatim}
有了这样的方法论视角,我们再谈论一些具体问题。

\section{程序员是无产阶级吗?}
\subsection{是否拥有生产资料这点是决定性的吗?}
左圈传统话语定义无产阶级,就看是否拥有生产资料。这个断语在许多场景下是“方便”的,但并非本质。无产阶级这个概念的价值所在,是这个阶级“能干什么”,“会干什么”;为了方便圈定一些人“能干什么”,“会干什么”的范围,给出了一个拥有生产资料的概念。

与其说决定无产阶级的“产”是“生产资料”的产,不如说是“产品”的产。而决定一个劳动者是否拥有产品,取决于谁拥有组织生产和分配产品的权力;只是在传统农业社会和早期的工业社会,可以直接决定“产品”所有权的才是是否拥有“生产资料”。

有产者凭借生产资料私有制垄断了组织生产的权力,从而剥夺了劳动者享有和支配产品的权力。劳动者只能把自己的劳动力作为一种商品与生产管理者交换,从而使自己成为雇佣劳动者,才有机会参与劳动过程,获得被剥削而得生存的机会。

这使得劳动者成为劳动的对象而非主人,在生产过程中被异化,失去了经济的权力,从而失去了政治的权力,带来各种各样的苦难。另一方面劳动者未来可出售的劳动被量化成资本,作为政治和经济权力的标记物被资本家垄断,让资本家得以按资本主义固有的逐利逻辑去统治这个世界,带来各种不符合社会利益的困局。

所以关键在于谁拥有劳动的组织权力和分配权力,而不在于用什么手段。垄断生产资料只是获取组织和分配权的一个主要途径。

\subsection{程序员拥有劳动的组织权和分配权吗?}
程序员的直接产品是代码,代码用电脑生产,运行在各种设备上,从而产生效益。然而电脑也好,运行设备也好,这些生产资料都不是决定性的因素。拥有它们,不拥有它们,并不决定一个程序员的阶级属性。关键在于代码的所有权最终归谁所有。

举个例子。如果facebook的代码所有权全部归我,而我除此之外不名一文。我仍然可以靠facebook代码的价值从各种渠道获得贷款,购买运行它们所需的基础设施,一样可以提供给所有人facebook带来的服务。当然,代码被作为一种生产资料,又是另一个环节了。

然而facebook的代码所有权并不归我,不归属任何一个开发者,甚至不归属扎克伯格。而是归于管理facebook的资本方,董事会。这个结果不是仅靠垄断生产资料而实现的,而是靠整个资本主义社会秩序。

资产阶级庞大的政治组织,强大的国家机器,先进的统治和弹压技术,维护了这个社会秩序的强势运行。在这个秩序里,一个人未来的劳动成果决不能兑换其他人过去的劳动成果,只有用货币化的资本才允许预购劳动者未来的成果。

这个秩序从法律层面得到了保证,若没有使用货币化的资本在国家机器里注册公司,则无法对多人协作的复杂劳动拥有产权的保证。法律代表的是国家机器的意志,它后盾是司法,警察,军队,飞机和大炮。

这个秩序使得不掌握资本的绝大多数劳动者失去了组织未来劳动的权力。当然,包括被雇佣的程序员。劳动合同里通常注明一条,程序员生产的一切代码归“公司”所有。而公司除了法律,除了国家机器之外,其实并没有其它任何绝对有效的手段阻止员工自己使用这些代码。

\subsection{具体分析}
有了这样的理论准备,再看程序员的阶级属性,就明了了。

\subsubsection{程序员接私活或者小作坊外包工作}
掌握了代码的所有权,能把代码而非劳动力作为商品出售。这样的程序员是小生产者,类似农场主,不是无产阶级。

\subsubsection{程序员做为技术管理者}
参与到代码的生产管理中,得到了企业合伙人的身份,获取一部分实际股份。他们既是技术专家,更是依附资本的职业经理人,而后者才是矛盾的主要方面。

这部分程序员的主要职责其实不是生产代码,而是靠技术组织能力去确保对劳动的控制,使资本对产品的所有权体现在生产过程中。为何这么说呢?假设技术负责人是程序员团队自己推选的,对程序员团队的利益负责,那资方的谈判权就会变得很差;而技术负责人,实际上是由资方随时全权任免的,管理者认可资本对代码的所有权,远远超过管理者技术能力的重要性。

\subsubsection{普通程序员}
程序员虽然拥有电脑,拥有开发软件,也有权力购买服务器的使用权……但他既没有对劳动过程的决定权,生产的代码也完全不属于自己。他们是非常典型的雇佣劳动者,即是非常典型的无产阶级。

在普通程序员的工作中,马克思主义对资本主义生产规律一切精辟分析都得以体现,甚至比自动化越来越普遍的传统生产行业更典型。例如项目研发是纯按工时计,资本家的利润完全在于程序员创造的剩余价值中。他们有极大的动力去延长工作日(鼓励晚下班),无薪加班(周末加班调休),薪酬保密(限制谈判权),项目外包(劳务派遣),按项目需求扩大队伍,逼迫员工提升劳动效率(28制度末尾淘汰),增加劳动强度(让程序员在十几年的工作日中因过度劳动,创造二十几年的产品,却失去了剩下十几年劳动所需的健康状态)……

这类程序员,作为无产阶级,在生产过程中有实际的动机去改变旧的秩序。

\subsubsection{做开源项目的程序员}
这类程序员,在开源项目的开发过程中,不仅是作为无产阶级,而且是作为无产阶级先进生产力的代表。

首先程序员的开源社区,本身是在人类社会最先进的协作技术下诞生的。多人参与,分工,协作,迭代路径,抽象化,管理,分发……每一个环节都体现了社会化大生产的组织技术之先进。

而且优秀的开源项目,无一是靠个人独裁能发展壮大的,优秀的社区和专业的委员会必不可少,是生产领域民主化的伟大实践。

大量开源项目免费传播,使代码作为一种生产资料,放弃了私有制。人人可以复制,甚至人人可以改动,使得新技术的传播和迭代极为迅速。而劳动者通过某些类型的开源许可证,不仅获得了公有的生产资料(代码),还获得了对劳动产品的支配权。

不获利的开源社区,非但没有像资产阶级预言的那样导致生产停摆,反而使无数人获得了更强大的生产能力,创造了更多劳动产品,让人们的财富变得更丰富。许多技术产品也从租让生产资料的所有权坐收渔利,转变成提供持续的劳动服务获取报酬。甚至大资本控制的公司也认为开源社区对自己有利,微软,谷歌,facebook反而是一些大型开源项目的支持者。

所以说,在程序员的行业中,存在着最先进的无产阶级队伍,代表了最先进的社会化大生产的组织和管理能力,代表了生产资料公有制的发展方向,代表了个体劳动去商品化去资本化的方向,代表了生产领域大民主的探索方向,对资产阶级存在的合法性提出了挑战。不仅如此,开源社区还给程序员以劳动的愉悦,因为劳动终于掌握在了劳动者自己手中。

\subsubsection{具体的程序员是哪一种程序员?}
{\bfseries 上文说了四种程序员,小生产者,资本的经理人,雇佣劳动者,无产阶级先进性的代表。具体到某一个程序员,如何划分他的归属呢?

其实这种形而上学的划分毫无意义。一个程序员,可能拿着单位雇佣劳动的一部分工资,又享受着作为管理者得到较大份额的股份,一边偷偷做些价值不菲的私活,一边在开源社区为大家制造伟大的轮子。}

所以形而上学的“脑力无产者”定义毫无意义。只有在具体的时间,具体的条件,围绕着具体的问题如谁是敌人谁是朋友,这时辨证地探讨某个范围的程序员阶级属性,才是有意义的。

\section{程序员是小资吗?}
熟悉政治词汇的人都明白,“小资”是一个过渡滥用的词,语义非常庞杂以至于常常失去意义。

{\bfseries 其实我们常说的小资,包含三个区别很大的范畴:
\begin{inparaenum}
  \item 小生产;
  \item 布尔乔亚,小市民文化;
  \item 对大资本的依附性。
\end{inparaenum}}

\subsection{程序员是小生产吗?}
上文说过,程序员在拥有自己代码的所有权,交易权,作为独立的商品去市场交换时,这是典型的小生产。

而许多公司就是几个程序员搭伙创办,分享利益的外包团队,其实也是一种作坊式的小生产形态。

程序员小生产的一面是落后的。小生产的程序员,和社会化大生产的利益不一致。社会化大生产会使自己的劳动效率和质量失去竞争性。市场越成熟自己的利益空间就越小。

小生产和技术进步方向不一致。因为技术进步带来新的竞争成本,间接减少收益;而另一方面,技术进步又总是趋向于更大规模的 协作。

小生产和公有制利益不一致。小生产虽然非常希望得到免费的生产资料,但另一方面维护自己产品的私有权又非常重要。

所以程序员小生产的一面是落后于时代发展的。但他们往往和其他无产阶级程序员处在同一生态,也被大资本用各种方式盘剥和挤占利润,又有反对大资本的动机。

小生产在社会化大生产方面和无产阶级,资产阶级都对立。但在私有制的一面支持资产阶级,在诉求权力的一面支持无产阶级,从而体现出了摇摆性。

程序员的小生产是程序开发行业扩充迅速,尚未成熟的产物。这个阶层必然随着时代发展不断消亡,又不断以新的形态出现。但在斗争激烈的时候,他们自然会成为两个对立阶级共同争取的同盟对象。

\subsection{程序员是小资情调的小资吗?}
小资情调,或是小市民的德行,这些都属于文化范畴。是经济存在于社会生活中的反映。所以程序员是否小资,是个个人化的问题,需要看这个人自己的成长经历和生活环境。

一方面,许多程序员从小资大本营——高校教育中成长出来,自我关注,物化,精英主义,成功学,物质享受这些“小资元素”浸淫了整个青春期,对个人文化和价值观的影响必然是极大的。工作同事(尤其是产品经理)又多是小资情调,更加重了同化氛围。

最关键的是,程序员的薪酬普遍比其他劳动者高一截,收入接近金融行业。诞生了许多所谓“高净值人群”,是整个资本主义社会营造的小资文化的主要消费者。一个程序员在生活中变得小资是大概率事件。

另一方面,由于中国经济发展的不平衡,许多程序员家庭来自农村或发展中的城镇,了解普通劳动者生活的艰苦。也更多地接触社会存在的压迫和不公。他们和出身于所谓“中产阶级”的青年会有显著的观念差别。

然而无论程序员在生活中多么沉醉小资文化,在他们所处雇佣劳动的工作领域,眼见得是残酷竞争,剩余价值剥削和成功学的破产。少部分人上升成为资本家的经理人,或者因为巨大的成功成为新晋资本家;但更多的人会在雇佣劳动中接受劳动者世俗文化的洗礼。这种洗礼与小资文化浸淫的矢量如何相互作用,就是另一回事了。

\subsection{程序员依附大资本吗?}
我个人认为,当代的程序员,对大资本有非常严重的依赖。使得整个程序员群体,无论在生产领域显得多么无产,最终的政治取向却是偏向小资的。从几个角度来看:

\subsubsection{程序员在产业链中的位置}
程序员收入特别高,初学者的工资比许多传统行业资深工程师还高。这一部分是因为程序员还不够多,互联网技术发展太快;而更关键的是程序员的特点使得技术成果特别利于资本家在产业链的话语权,从而占有更多剩余价值。

典型的例子是滴滴这类打车软件。它虽然给出行和接车带来便利,但盈利点却不在于此。打车软件的盈利趋势还是一方面运作资金金融化,另一方面接管传统出租车行业。自营出租车是打车软件的普遍趋势。

在这些行业中,创造价值的劳动者,因为技术的进步反而日益失去话语权,对劳动价值的占有比例只会日益下降。这个趋势显然利于资本方,正因如此,资本方让程序的开发者在产业链条中的一个很高位置上,占有的分配份额远比底层劳动者多。

而维持这个秩序的主要是资本方的利润法则。程序员如果自觉拥护这个法则——比如潜移默化地认为自己复制粘贴看不懂的代码,创造的劳动价值比其他工人、工程师都要高;或是自己架构个web网站所需要的天赋,远远超过别人架构一个大型产业或公共工程——就会自觉去支持创造这个秩序的大资本一方。

\subsubsection{程序员高收入换取的金融化资产}
许多程序员因为收入颇丰,存款不少;于是被无数资本掮客,比如余额宝带头的p2p金融引入了金融行业,或是购买理财产品,或是购买股票等有价证券,或是购买了房产。然后期待增值。

以资本的年化收益率百分之N计算,利息收入超过工资所得,需要奋斗K年。这个换算的结论就是几年实现所谓财务自由。而财务自由,指的是“过上梦寐以求的不劳而获的生活”。这一方面体现了资产阶级意识形态,而另一方面也利用了雇佣劳动者对异化劳动的痛恨。

凡是做着金融资产收益匹敌劳动所得美梦的程序员,都极大地依赖大资本。他们希望大资本在股市里赚钱,希望房地产涨价。他们的利益与买空卖空无关,与其他人的穷困无关,与国际秩序的和平稳定无关。越是符合大资本利益,自己就越能享受到资产账面增值的愉悦感。

这类人有时不仅是大资本的附庸,在社会变革导致大资本利益严重受损时,他们会转变为维护大资本利益的主力。在这个阶段,因为金融资产依附大资本的人是最反动的红脖子,极右选民,种族主义者……

而他们不知道的是,越来越多的人从金融收益中甩掉了劳动,谁来被盘剥养活他们呢?所谓高净值人群,本质上只是大资本的小猪存钱罐罢了。平时把小钱分散存在小资产阶级的手里,需要时把存钱罐全都砸了,取钱走人,留下一地碎片。

这类剪羊毛的事情且不论改革开放以来搞了多少次,就最近两年也够很多人哀唱“从头再来”了。

程序员作为产业链上游的高收入群体,通过金融资产对资本的依附是很明显的。

\subsubsection{程序员行业对大资本的依附}
程序员之所以收入高,是行业效益好,工作不愁。尤其是最近几年的互联网行业。然而背景并非互联网行业每个角落都探索出了更高效的生产方式取代传统行业。而是“投资驱动”四个字。

从06年金融危机以来,尤其是08年金融领域开闸放水再也关不住之后,互联网行业才跟着兴盛起来。无数企业朝生夕死,支撑他们的是投资,投资,投资。

天使轮投资撑半年,做个无用的产品融资A轮。A轮融资做个花架子融B轮,B轮融资熬完互联网泡沫都快破了,其中能捞多少捞多少。

许多程序员的高工资,就是在这个泡沫投资的生态中被吹泡泡吹起来的。从这个角度来看,程序员无论有意识,无意识,都是金融资本的依附者。一旦金融资本的危机爆发,程序员们欠的房贷,拖的房租,没还的信用卡,都成了勒住脖子的绳索。

程序员不管有意识,无意识,只要满足于自己的高薪待遇,就必须和大资本的利益一致。这种依附在过去几年发展得很重,而在当前经济形势下可能要检验效果了。

\subsubsection{程序员的“上升空间”}
\begin{note}
  因为不睡觉这么久终于开始犯困,所以这些浅而易见的道理就不写了。
\end{note}

\section{结论}
从马列主义的角度看待程序员的阶级归属,并非“是否掌握生产资料”这么个形而上学的命题。必须用辩证法的思维设定各种条件,并在具体的价值判断(在何时的什么情况下,谁是谁的敌人,谁是谁的朋友)中才有意义。左青围绕着“领导权”对“脑无”先进性和原罪的形而上学讨论是无聊的。

程序员是一个庞大而复杂的群体。应该说,程序员的生态中,最本质的矛盾(生产领域)的主要方面是偏向无产阶级的,而且一部分还代表最先进的社会主义生产力发展方向。

{\bfseries 但在实际生活中,因为文化的小资化,和对大资本的依附,我认为当代的程序员整体更偏向于小资,在未知的社会变革突现的时候,首先是以维护大资本利益的保守面貌出现,只有等梦幻破产了之后才会和传统的无产阶级队伍融合,逐步体现出无产阶级的先进性。}

\chapter{应该清楚男友哪些方面才能结婚?}
今天晚上,得知我最好的朋友离婚了。孩子才1岁,同样身为父亲的我感到难过,离婚对孩子来说影响太大。

我没资格说这些。不过,至少我的婚姻算凑合,我也有把握延续这种凑合。我写这篇文章,就是希望有更多的人能把握自己的婚姻。就算我说的都是废话,如果能引起朋友们在婚前对婚姻多一些思考,就达到目的了。

每个人对婚姻的态度都是不同的,随便找个人过日子的,找个女人帮自己生孩的,找个过夜不收钱的,这些所谓的“婚姻”就不在讨论范围了,这里和大家讨论的,是高质量的婚姻。

每个人的理解都不一样,在我看来,婚姻的状态有很多种,\textbf{但是让人舒服的婚姻,一定都是有爱的。}

有人说,爱情是有保质期的。我说,有保质期的不叫爱情。那叫激情,激情夹杂的东西太多,性欲,感动,内疚,憧憬,有太多太多的杂质,这样的情感确实难以持久。何况,激情往往是精心呵护起来的,一旦丢失了精心呵护的动力,褪色太快。

泡妞的时候,激情是最好的工具。然而面临结婚选择时,作为男人,一定要理清自己的头脑,祛除激情的成分。这个思考的过程非常重要,婚姻是没有回头路的。别以为大不了还可以离婚。离婚不是解脱,是又一个麻烦的开始。

如何确定自己爱不爱一个女人,这是非常关键的一步。我很肯定的说,很少男人清楚这个问题。男人的生理天性决定了对女人的选择很大因素是外貌。但是婚姻是反人类天性的。所以当男人选择了婚姻,就注定要克服自己的生理天性。

男人的生理天性,说白就是希望保质保量的遗传自己的基因,比如尽可能的和更多女人上床,尽可能的选择更漂亮,基因更优秀的女人上床。

而婚姻,恰恰相反,男人只能选择一个女人。而且要知道,眼前这个将要嫁给自己的美女,不几年就会生孩子,眼角会有皱纹,乳房会下垂,乳头会变的很黑,屁股会变形,小肚子会出来,还会有难看的斑纹。更可怕的是,男人的生理天性决定了男人老和同一个女人上床,是会腻的。所以,没有爱的婚姻是危险的。因为没有克服生理欲望的信念。

兄弟们永远别在婚前信誓旦旦的说自己不是那种包二奶的人。50年代出身那批人,道德观念比我们重的多,老婆还都是共过患难的糟糠,不一样大量的出轨?所以,在一夫一妻的制度下,保障婚姻的根本,还是得依靠爱情。其他约束在人性本能的拉扯下都是脆弱的。

据我观察,很多失败婚姻的元凶就是性。性对男人的诱惑是致命的,不少男人就是因为和女人上了床就顺理成章的结婚了。是好是坏全看运气。压根不思考有没有共同语言,遇到问题双方的沟通方式彼此能否接受,这些关键的问题不去想。就因为上了床就结婚。这是很多悲剧的来源。用人性本能来适应反人性本能的婚姻制度。这是非常愚蠢的。更愚蠢的是很多女人以性为工具来对待婚姻。说到这里才发现,婚姻真是个大话题。每个人去领结婚证的时候,都没想过有一天会去拿离婚证。可惜,离婚的人越来越多。据统计80后的离婚率快到百分四十了。(见文末离婚率说明)

男人,只有把性欲,感动等等等等因素全抛开,才能真正认识自己对一个女人的感情是不是爱。爱不能解决所有问题,但是能给男人解决问题的动力。没有动力经营的婚姻,麻烦很多。因为婚姻的问题很多。

结婚对男人而言,纯粹是责任。有了承担责任的决心,才敢谈婚姻,所以,结婚一定要找一个自己爱的女人。自己余下的生命全部为之付出的家庭,女主人不是自己真正爱的,太悲剧了。因为婚姻唯一能回报给男人的,仅仅是一只能牵着自己走向死亡的手。千万别奢望婚姻回报给自己什么,任何奢望只会添加双方的压力。做好丈夫该做的,自然得到应得的。

很重要的一点,不是平时有没有共同语言,而是有分歧和问题的沟通。说白了,如果男人是个讲道理的人,女人也得是个懂道理的人,如果男人是个喜欢用拳头说话的人,女人也得是个肯挨拳头的人。女人如果喜欢唠叨,男人就得听得惯。总之,两个人必须得有个拿出统一意见的程序,且双方都能接受。最好是都乐意接受。其实这一条对于真正相爱的人来讲不是问题,真正相爱的两个人永远是站在对方立场思考问题的。

有情饮水饱,只能饱一顿。男人没结婚,是潇洒的,一结婚,负担就来了。两个人的结合所带的压力肯定是巨大的,必须要考虑双方的承受能力,自己没有能力负担,就不要害人害己了。女朋友如果喜欢钱,有钱就娶,没钱就不要砸锅卖铁娶回家了。

有些负担,背上就是一辈子的。对婚姻而言,肯陪自己一起奋斗的女人,才真正应该珍惜。单单是两个人一起改善生活的过程,就已经是婚姻宝贵的财富了。别说现在这样的女人少,不少。老盯着花枝招展想靠嫁人改变命运的年轻女人,就别怪女人现实。自己从未规划过未来,就别怪女人不肯陪自己一起奋斗。

我一直提倡婚前性行为,性生活得到满足的男人,才更容易发现女人其他的魅力。而现实的情况是,不少好女人还剩着,男人却在追逐性的过程中迷失了自己。因色而结合,女人色衰之后,没本事的男人继续鬼混,有本事的换个女人凑合。这样的风气还有个很烦的影响,女人越来越在意自己的形象,化妆时间越来越长,衣领越来越低,裙子越来越短。而充电的时间越来越少,独立意识越来越淡。

男人一边追逐女色,一边抱怨女人素质越来越低。

女人呢?一边抱怨男人肤浅,一边不断迎合肤浅的审美。

婚姻不是人生的全部,两个人携手一生,相助则利,相阻则损。性格互补也好,志同道合也好,彼此成为对方的助力,真的很重要。如果和一个女人婚前就感觉疲惫。相信我,别结婚了。\textbf{家是港湾,是一个让男人快到家门就会不自觉加快脚步的地方。是一个一回家就会彻底放松,卸下所有伪装的地方。是一个所有笑容都发自真心的地方。}

怎样区分激情和爱情?其实很好区分的,当你不见她时茶饭不思,是激情。老想见到她,想和她一起玩,一起上床,一起做白日梦,是激情。每天短信电话多的没完,是激情。为了生日节日费尽心思想浪漫的点子,是激情。

当你和她在一起时脑子里不自觉的规划实实在在的将来,是爱情。当你们争吵到很凶,火很大时,也不忍心说一句伤害她的话,是爱情。当你们有分歧时,你总是能清楚的知道她是怎么想的,能理解她的初衷,是爱情。

现在很多人谈恋爱很短时间就结婚了,当然,不是说时间短就不好,而是要清楚的知道,婚前的考察对这一生的影响。失败的家庭是任何成功都弥补不了的。

人,一旦离过一次婚,再寻找幸福的难度就更大了。因为,离过婚的人更难相信爱情。更何况离婚对子女的影响,太大太大。人,真正能留在这个世界上的,也就是子女这点血脉而已。为子女营造一个好的成长环境,是父亲的责任。 

前面说这么多其实都是说面对婚姻的态度要理性,要慎重。  态度端正以后,技巧也有很多需要注意的地方。明明相爱的两个人互相伤害的事情太多了。如果犯错,也要积极想办法补救争取。伤寒心散伙的也多,而且婚姻生活,男人还有个重要技术要掌握,就是老妈与老婆的关系。婚后生活琐碎的事情太多,先说说处理婆媳关系。所有还没结婚的兄弟一定不要把这不当一回事。一边是生你养你为你付出所有的母亲,一边是将要相伴你一生的老婆。双方都有责任要照顾。这个关系弄顺了,少很多麻烦。 

有人说男人在婆媳之间是双面胶,夹在中间两面讨好。这是错的。男人必须是一手拿棒一手拿润滑油。小事居中调节,原则问题对事不对人,道理绝不能歪,既不能纵容媳妇,也不能惯坏了老妈。坚决不要在老妈面前说媳妇不对,多包涵之类,也不能在媳妇面前说,这都是纵容。把握一个原则,绝不允许媳妇在自己面前说老妈的坏话,有事说事。该怎么解决怎么解决,是老妈不对也要和老妈把事情讲清楚,同样的,老妈说自己媳妇的时候也要分清是非。居中讨好的后果是两面不是人,矛盾还越来越深。 

总之,绝对的公平是处理婆媳关系的唯一方式,委屈任何一方,都会使矛盾激化。妈也好,老婆也好,底线都给她们画好。人都是有选择性的,既然要不到特权,自然会注意彼此相处。一开始可能男人日子不太好过,两边都要斗上几次。但是慢慢的,家庭秩序就会走上正轨。 如果一开始图省心,哄过去,两边脾气都养大了就够自己受了。(前面忘了说了,找女人一定要找个聪明的,笨女人多很多麻烦,聪明的女人自己会处理,自己只用打打下手,关系就处的很好了。) 

还有个问题,就是子女,现在的孩子才真叫一个宝贝,一大家人围着转,各有各的主意,很多矛盾都是因此而起,在这一点上,男人一定要强硬,从一开始就绝对不能让步,要给出明确的信号,孩子是自己的。双方父母肯帮忙,感谢。但是涉及孩子的一切决定。必须是自己拿主意。(这一点背地里可以多听老婆的,毕竟,妈妈是最爱孩子的。)这一点申明尤其重要,相当于自己一个人把所有炮火揽了。否则,孩子一点点小感冒家里可以闹翻天。(我那今天离婚的朋友就是因为这个)。

婚姻确实是个大话题,一时反而不知道说什么好了,我是真心想和大家交流一下。因为我自认为自己的婚姻真的是种幸福。我是真心希望有更多的人能愉快的享受婚姻。有人说不吵架不算夫妻,说真的,我和我老婆真还吵不起来架,就像我前面说的,伤彼此的话确实不忍心讲出口。何况一旦清楚彼此都是出于爱,又有什么好吵的?有几次刚进入状态,看见彼此装腔作势生气的样子就都笑了。 

512地震的时候,我在震区,这么大的城市,通讯中断的情况下,我第一时间见到了我的老婆,不在我们上班的附近,也不在我们家附近,而是在我父母的家门口。她知道体谅我担心父母的心情,我知道她能体谅我。这就是婚姻的默契。也是婚姻和恋人的区别,婚姻承载的更多。 

恋人时刻只有甜蜜和浪漫,而婚姻则更多是责任和平淡,这个转化的过程,是需要双方有充足思想准备的。其实只要两个人肯一起面对,婚姻生活平淡中的幸福并不输给热恋时的浪漫。

愿意花时间看这帖子的兄弟姐妹们,光是这一点就足以说明你们对待婚姻的态度是充满诚意的。有这样的态度你们一定会收获属于你们的幸福。真心话。
\end{document}